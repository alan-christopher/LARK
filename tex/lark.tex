\documentclass[12pt]{article}

\usepackage{booktabs} \usepackage[table]{xcolor} \usepackage{titling}

\title{Lightweight Adventure Roleplay Kit}
\begin{document}

% Minimal title.
\setlength{\droptitle}{-100pt}
\predate{}
\date{}
\postdate{}
\preauthor{}
\author{}
\postauthor{}
\maketitle

\section{The Fundamental Rule of LARK}

Dungeon masters, or DMs, build the worlds of tabletop RPGs and
roleplay them during the game. The fundamental rule of LARK is this:
\emph{\textbf{If the DM says it is so, then it is.}}

DMs, this means that the only real limitation to what you can do with
the world is your imagination; swords-and-horses fantasy is as much an
option as space opera, or hard sci-fi, or whatever else. However, your
unfettered powers mean you need to be careful -- more careful than
usual -- to avoid appearing arbitrary or capricious. It doesn't take
long for a DM with unhappy players to become a DM without players.

\section{Player Actions} Players may attempt any action in-game, but
success is not guaranteed. To determine the consequences of an action
the DM first judges its difficulty, taking into account the player
character's background and circumstances (e.g. it may be easy for a
seasoned archer to hit a target at 30 pace, but for a child even
drawing the bow may be impossible). The DM may decide to share the
difficulty of the action, or at least the player character's
perception of the action's difficulty, with the player. Next, the
player rolls a six sided die until rolling a 1 or meeting a threshold
specified by the DM (e.g. ``Give me a run of length 2''). Lastly, the
number or rolls \textbf{before} the 1 is compared to the table below:

\begin{center} \rowcolors{2}{gray!25}{white}
  \begin{tabular}{lr}
    \toprule
    \textbf{Difficulty} & \textbf{Run Length} \\ \midrule
    Trivial       & 0 \\
    Easy          & 1 \\
    Moderate      & 2 \\
    Hard          & 4 \\
    Heroic        & 8 \\
    Inconceivable & 16 \\
    Impossible    & $\infty$ \\ \bottomrule
  \end{tabular}
\end{center}

If the run length meets the difficulty criterion it succeeded. If it
hit the next difficulty up, it succeeded wonderfully, and if it only
reached the difficulty below it failed spectacularly.

\end{document}
